\subsection*{Problem 22-2 Articulation points, bridges, and biconnected components}
\begin{enumerate}
	\item	若$G_{\pi}$的root只有一个儿子,在删除root之后,图的连通度不会增加,即不是articulation points;若$G_{\pi}$的root有多于两个儿子,在删除root之后,图的连通度会增加,即$G_{\pi}$是articulation points
	\item	若$v$为articulation points,则$v$的儿子中必然有一个儿子,使得其没有back edge连接到$v$的祖先;否则若$v$没有这样的儿子,由于是无向图,当删除结点$v$之后,图依然保持连通,即$v$不是articulation points;反之亦然
	\item	利用DFS,可以在$\mathcal{O}(V + E) = \mathcal{O}(E)$的时间内维护每个结点的$low$值
	\item	Vetex $v$ is an articulation points iff. ($\attrib{v}{low} \geq \attrib{v}{d}$) or ($v$ is the root of $G_{\pi}$ and $v$ has at least two children in $G_{\pi}$). The running time of the algorithm is $\mathcal{O}(V + E) = \mathcal{O}(E)$
	\item	设边$(u, v)$是图$G$的bridge,若边$(u, v)$在一个图$G$的一个simple cycle中,那么当删除边$(u, v)$之后,必然存在一条$u$到$v$的path,使得$u$和$v$依然保持连通,与边$(u, v)$是bridge矛盾,所以边$(u, v)$必然不在一个simple cycle之中;反之亦然
	\item	Edge $e = (u, v)$ is a bridge iff. ($e$ is a tree edge) and ($\attrib{u}{d} < \attrib{v}{low}$). The running time of the algorithm is $\mathcal{O}(V + E) = \mathcal{O}(E)$
	\item	Equivalence relation: $e_1 \sim e_2$ iff. ($e_1 = e_2$) or ($e_1$ and $e_2$ lie on some common simple cycle). Therefore, the biconnected components of $G$ partition the nonbridge edges of $G$.
	\item	在DFS过程中,利用一个stack来存放所访问的边,若遇到articulation point或bridge,则将之后的所有的边pop,并将bcc标记成相同标号,运行时间$\mathcal{O}(V + E) = \mathcal{O}(E)$
\end{enumerate}

