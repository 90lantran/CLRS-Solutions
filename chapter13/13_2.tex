\subsection*{Problem 13-2 Join operation on red-black trees}
\begin{enumerate}
	\item	(Omit!)
	\item	因为$\attrib{T_1}{bh} \geq \attrib{T_2}{bh}$,根据Red-Black Tree的性质,从$T_1$的Root开始,若有Right-Child,则向Right-Child走,否则向Left-Child走,必然存在一个Black节点$y$使得$\attrib{y}{bh} = \attrib{T_2}{bh}$,且保证$y$是所有满足条件的节点中key最大的 \\
		算法的时间复杂度:$\mathcal{O}(\log{n})$
	\item	算法时间复杂度:$\mathcal{O}(1)$ \\
		\begin{codebox}
		\Procname{$\proc{RB-Join'}(T_y, x, T_2)$}
		\li	$\attrib{z}{left} \gets T_y$
		\li	$\attrib{z}{right} \gets T_2$ 
		\li	$\attrib{z}{parent} \gets \attrib{T_y}{parent}$
		\li	$\attrib{z}{key} \gets x$
		\li	\If $\attrib{\attrib{T_y}{parent}}{left} = T_y$
		\li	\Then
				$\attrib{\attrib{T_y}{parent}}{left} \gets z$
		\li	\Else
				$\attrib{\attrib{T_y}{parent}}{right} \gets z$
		\end{codebox}
	\item	Red \\
		因为树$T_y, T_2$的Root的颜色都为Black,如果$x$的parent的颜色也为Red,则将$x$的颜色置为Black,且将$T_y, T_2$的Root的颜色置为Red,递归向下调整,保证Black-Height不变 \\
		时间复杂度:$\mathcal{O}(\log{n})$
	\item	同理,若$\attrib{T_1}{bh} \leq \attrib{T_2}{bh}$,则用上述算法对称处理
	\item	The running time of $\proc{RB-Join}$ is $\mathcal{O}(\log{n})$
\end{enumerate}

